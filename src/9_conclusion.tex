\chapter{おわりに}
\label{conclusion}

\section{本研究のまとめ}
本研究では骨格推定を用いた音声フィードバックシステムでボディビルのポージング練習を支援するシステムを提案した。
本システムを用いてポージング練習を行いポーズの改善、そしてその効果の保持の検証、そして既存の練習方法である鏡を利用したポージング練習との比較を行った。
その結果、本システムを利用してポーズを改善された被験者がいたものの統計的に有意な結果は得られなかった。効果の保持に関しても鏡を利用した練習と比較して有意な差は認められなかったものの、システムを利用することでも鏡と同様の効果の保持が考えられる結果となった。

\section{今後の展望}

本研究では骨格推定を用いた音声フィードバックシステムの有用性について十分に検証することができなかった。
そのため、学習が定着する練習量を検証する予備実験などを行い、その後、よりサンプル数の多い実験の実施や、練習期間を十分にとった検証を行なっていきたいと考えている。
また、本システムは実験の時間の短縮、被験者の負担軽減のために肘、肩に対してのみ、平面的なフィードバックを行ったが、ボディビルにおけるポージングは全身のバランスも重要な要素である。
よりボディビルでの有用性を高めるためにも、全身に対して奥行きも考慮したフィードバックを行うことができるシステムを構築していきたい。
今回の実験ではシステム利用者全員に対して同一の角度を理想のポーズとして設定したが、本来利用者それぞれの体格や筋肉量、ボディビル経験などよって理想のポーズは異なる。
そのため、利用者本人が理想とするポーズを設定できるよう、理想のポーズを画像で入力できるといったようなシステムを構築することでより利用者に寄り添ったシステムを作っていきたいと考える。
今回は骨格推定の方法として既存のモデルを利用したが、筋肉量が多いボディビルダーでは骨格推定ができない、精度が下がることもあるため、ボディビルダー等の筋肉量が多い人でも精度高く検出できるようなモデルの構築についても検討したい。

本研究での骨格推定を用いた音声フィードバックシステムはボディビルだけではなく他の様々なスポーツ、フィットネスの現場においても応用が可能であると考えられるため、
まずはボディビルという領域で有用性を示し、その後他の領域にも応用していきたいと考えている。

本システムの有用性を検証できれば、初心者が手軽にポージングの練習ができるようになり、ボディビル競技の普及、競技レベルの向上が考えられる。
また、音声のフィードバックを用いることで視覚障害者でもポージング練習を単独でできるようになることも期待でき、ボディビルのバリアフリー化にも貢献できると考えられるため、
今後の研究によってより多くの人に利用してもらえるようなシステムを改善していきたい。
