\chapter{考察}
\label{consideration}
本研究では本システムのポーズ改善の効果の検証と既存の練習方法である鏡を利用した練習との効果の差についてにについて検証したものであるが
今回の実験では本システムの利用によって練習直後のポーズの改善が見られた被験者がいたものの統計的に有意とは言い難い結果となった。(pose1: p=0.8782,pose2: p=0.07812)
また、24時間後のポーズの改善についても統計的に有意とは言い難い結果となった。(pose1: p=0.4674,pose2: p=0.8125)
練習方法の比較では鏡を利用した練習群と本システムを利用した群で練習後、練習後24時間後共に練習効果の有意な差は認められなかった。

\section{本システムのポーズ改善への効果}
\label{sec:pose_improvement_from_system}
今回の実験で統計的に有意な結果が出なかった理由としては、3つ考えられる。
1つ目は被験者の数の不足。
今回の実験では14名の被験者を用い、pose1,pose2それぞれで7名ずつ本システムを利用して練習を行ったが、サンプルに偏りがあった可能性が考えられる。
2つ目は練習の量が不足である。
ボディビルのポージング練習はトップ選手でも高頻度で30分程度行うことが多くあるが、今回は実験スケジュール、被験者の負担軽減のために
1ポーズあたり10分(30秒練習、30秒休憩を10セット)のみの練習であり、ボディビルのポージング経験のない被験者のみだったためポージングに慣れる前に練習が終了してしまった可能性が考えられる。
これは、既存の練習方法である鏡を利用した練習の群でも同様に練習の有意な効果がみとめられなかったことからも考えられる。
3つ目はフィードバックの方法である。
本システムでは理想のポーズとの関節角度の差に応じて「肘を曲げる」、「肘を伸ばす」といったシンプルなフィードバックを行ったが、
このようなシンプルなフィードバックを行った意図としては被験者本人が練習する過程で曲げ伸ばしを繰り返して理想の角度に近づいていくことで内在的フィードバックを獲得し練習の効果の保持を期待したためであったが、
ポージングの経験のない被験者というターゲットに対しては「~度曲げてください」といった定量的なフィードバックを行った方が効果的であった可能性も考えられる。

\section{練習方法の違い}
本システムを利用した群と鏡を利用した群の比較において練習後、練習後24時間後共に練習効果の有意な差は認められなかった理由としては
\ref{sec:pose_improvement_from_system}で述べたことと同様に被験者数、練習量の不足などが考えられるが、
pose1の練習後24時間後の結果においてp=1となったことから、保持テストにおいては練習方法によって改善の効果に差がない可能性も考えられる。
仮に本システムが鏡を利用した練習と同等の効果があるとするとポージング練習において全身が映る鏡を用意する必要がない本システムを利用することで
初心者でも手軽に自宅等でポージング練習ができるようになり、ポージング練習のハードルが下がると考えられる。