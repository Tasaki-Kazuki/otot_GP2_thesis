\chapter{序論}
\label{introduction}

\section{はじめに}
\label{introduction:background}
ボディビル大会への出場は敷居が高く、トレーニング、減量だけでなくステージでの見栄えを良くするためにポージング練習も必須となる。
ポージング練習は初心者単独で行うのは難しく、トレーナーに指導を受けるという方法があるが高額である。
本研究では、骨格推定ライブラリを用いてカメラの入力から理想のポーズとの関節角度を比較し、音声フィードバックを返すシステムを構築した。

筆者は2020年から慶應のバーベルクラブというボディビルをはじめとしたフィットネスの大会に出場するサークルに所属し、ボディビル競技を行っている。
トレーニングや減量に関しては、YouTubeやSNSなどで学び、試すことで使用重量や体重の変化を実感することができた。ポージングについても同じように勉強し練習したが、こちらは定量的に変化がわかるものではなかったため苦戦した。
そこで他者に指導を受けることも考えたが、2020年からの新型コロナウィルスの影響で他者とトレーニングを行うことやポージングレッスンに参加することが困難であった。
ボディビル大会への出場経験がない中での単独でのポージング練習はステージで評価されるようなポーズへ近づいているという練習の効果を実感することが難しかった。

そのような理由から初心者が単独でポージング練習ができ、ポーズを獲得できるようなツールを作成したいと考えた。


本研究では、骨格推定ライブラリであるMediaPipe Pose \cite{mediapipe_pose_landmarker}を用いたポージング練習支援システムを提案し、ポーズ獲得に有効であるかを検証した。
骨格推定を用いたポージング練習の手法の確立はボディビル競技者の単独でのポージング練習におけるコストや時間、環境などに対する問題を解決し、ボディビル以外のポーズやフォームを重要とするスポーツへの活用へとつながると考える。


