\chapter{序論}
\label{introduction}

\section{はじめに}
\label{introduction:background}
ボディビルをはじめとするフィットネス大会に出場する人は増加傾向にある。日本ボディビル・フィットネス連盟(JBBF)の登録選手数は2015年の2213人から2022年の5701人へと2倍位以上に増加している\cite{jbbf}。
しかしながら、ボディビル大会への出場は敷居が高く、トレーニング、減量だけでなくステージでの見栄えを良くするためにポージング練習も必須となる。
ポージング練習は初心者単独で行うのは難しく、トレーナーに指導を受けるという方法があるが高額である。
本研究では、骨格推定ライブラリであるOpenPoseを用いてカメラの入力から理想のポーズとの関節角度を比較し、音声フィードバックを返すシステムを構築した。
% ボディビルをはじめとするフィットネス大会に出場する人は増加傾向にある。日本ボディビル・フィットネス連盟(JBBF)の登録選手数は2015年の2213人から2022年の5701人へと2倍位以上に増加している\cite{jbbf}。
% しかしながら、ボディビルではトレーニングや、減量、日焼け、ポージングなどやることが多く、初心者にはハードルが高い部分がある。

筆者は2020年から慶應のバーベルクラブというボディビルをはじめとしたフィットネスの大会に出場するサークルに所属し、ボディビル競技を行っている。トレーニングや減量に関しては、YouTubeなどを用いてステージでの見栄えを良くするためにポージング練習も必須である。しかしながら,2020年は新型コロナウィルスの影響で他者とトレーニングを行うことやポージングレッスンに参加することが困難であった。そのため、自宅でのポージング練習を行うことが多くなった。ボディビル大会への出場経験がない中での単独でのポージング練習はステージで評価されるようなポーズへ近づいているという練習の効果を実感することが難しかった。
% 練習方法としては鏡を使った練習や動画、写真に撮ることで見返すという方法をとっていた。しかし、一般的な家庭にあるサイズの鏡ではポーズをとった時に全身が映らなかったり、全身を俯瞰して見ることが難し買ったりといった問題があった。写真や動画に撮る方法では俯瞰して見ることや背中側のポーズを確認するといったことは可能だったが、ポーズをとるごとに見返す手間があった。

そのような理由から初心者が単独でポージング練習ができ、ポーズを獲得できるようなツールを作成したいと考えた。


本研究では、骨格推定ライブラリであるMediaPipe Pose \cite{mediapipe_pose_landmarker}を用いたポージング練習支援システムを提案し、ポーズ獲得に有効であるかを検証する。
骨格推定を用いたポージング練習の手法の確立はボディビル競技者の単独でのポージング練習におけるコストや時間、環境などに対する問題を解決し、ボディビル以外のポーズやフォームを重要とするスポーツへの活用へとつながると考える。


