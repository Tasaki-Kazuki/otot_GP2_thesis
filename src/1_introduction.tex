\chapter{序論}
\label{introduction}

\section{はじめに}
\label{introduction:background}
ボディビルをはじめとするフィットネス大会に出場する人は増加傾向にある。日本ボディビル・フィットネス連盟(JBBF)の登録選手数は2015年の2213人から2021年の5576人へと2倍位以上に増加している\cite{jbbf}。
しかしながら、ボディビル大会への出場は敷居が高く、トレーニング、減量だけでなくステージでの見栄えを良くするためにポージング練習も必須となる。
ポージング練習は初心者一人で行うのは難しく、トレーナーに指導を受けるという方法があるが高額である。
本研究では、骨格推定ライブラリであるOpenPoseを用いてカメラの入力から理想のポーズとの関節角度を比較し、音声フィードバックを返すシステムを構築した。


\cite{fitness}

% \section{本論文の構成}

% 本論文における以降の構成は次の通りである.

% ~\ref{background}章では,背景を述べる.
% ~\ref{issue}章では,本研究における問題の定義と,解決するための要件の整理を行う.
% ~\ref{proposed}章では,本研究の提案手法を述べる.
% ~\ref{implementation}章では,~\ref{proposed}章で述べたシステムの実装について述べる.
% ~\ref{evaluation}章では,\ref{issue}章で求められた課題に対しての評価を行い,考察する.
% ~\ref{conclusion}章では,本研究のまとめと今後の課題についてまとめる.


%%% Local Variables:
%%% mode: japanese-latex
%%% TeX-master: "../thesis"
%%% End:
