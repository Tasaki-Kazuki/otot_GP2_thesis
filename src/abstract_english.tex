Abstract of Bachelor's Thesis - Academic Year 2023
\begin{center}
\begin{large}
\begin{tabular}{|p{0.97\linewidth}|}
    \hline
      \etitle \\
    \hline
\end{tabular}
\end{large}
\end{center}

~ \\
The number of people competing in bodybuilding and other fitness competitions is on the rise. The number of registered competitors in the Japan Bodybuilding and Fitness Federation (JBBF) has more than doubled from 2,213 in 2015 to 5,576 in 2021.\cite{jbbf}
Success in bodybuilding competitions depends on a variety of important factors, including weight training, posing skills, and weight loss.

Of the various elements, posing is particularly challenging for beginners to learn on their own. One of the main obstacles for beginners in learning posing practice is that it is difficult to practice alone.
In this case, they may consider receiving instruction from others, such as a personal trainer, but in many cases, the cost is prohibitive and not all beginners have access to such instruction.
To solve this problem, we have developed a posing practice tool that uses skeletal estimation technology.

This system uses a skeletal estimation library called MediaPipe Pose to recognize the user's pose from the camera input and compare the joint angles with the ideal pose.
The system provides real-time audio feedback on the difference in joint angles from the ideal pose, and we believed that the system could enable beginners to improve their posing skills and acquire poses on their own.

The results of the experiment showed that although some subjects were able to improve their poses by using the system, it was not statistically superior.
Comparison of practice methods also failed to show a statistically significant difference compared to practice with mirrors, so we plan to increase the number of samples and the amount of practice, and to improve the system for future verification. If the usefulness of this system can be demonstrated, the results of this study are expected to help reduce the burden on beginners in bodybuilding posing practice.

  ~ \\
Keywords : \\
\underline{1. Bodybuilding},
\underline{2. Posing},
\underline{3. Pose estimation},
\begin{flushright}
\edept \\
\eauthor
\end{flushright}
