Abstract of Bachelor's Thesis - Academic Year 2023
\begin{center}
\begin{large}
\begin{tabular}{|p{0.97\linewidth}|}
    \hline
      \etitle \\
    \hline
\end{tabular}
\end{large}
\end{center}

~ \\
The number of people competing in bodybuilding and other fitness competitions is on the rise. The number of registered competitors in the Japan Bodybuilding and Fitness Federation (JBBF) has more than doubled from 2,213 in 2015 to 5,576 in 2021.\cite{jbbf}
  However, success in bodybuilding competitions depends on a variety of important factors, including weight training, posing skills, and weight loss.

  However, posing is challenging for beginners to learn on their own. One of the main obstacles for beginners in learning posing is the difficulty of practicing alone.
  Instruction by a personal trainer can be helpful, but is often expensive and not available to all beginners.
  To solve this problem, I developed a posing practice tool that uses skeletal estimation technology.

  This system uses a skeletal estimation library called MediaPipe Pose to recognize the user's pose from the camera input and compare the joint angles with the ideal pose.
  The system provides real-time audio feedback on the difference in joint angles from the ideal pose, and we believed it would enable beginners to improve their skills and acquire poses on their own.

  TODO ここに実験の結果とか締め的なことを書く
~ \\
Keywords : \\
\underline{1. Bodybuilding},
\underline{2. Posing},
\underline{3. Pose estimation},
\begin{flushright}
\edept \\
\eauthor
\end{flushright}
