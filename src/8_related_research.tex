\chapter{関連研究}
\label{related_reserch}

武蔵野大学の鎌田らは \cite{Relatedresearch1}スクワットのフォームに対してOpenPoseを用いて姿勢差分に用いる関節角度の抽出方法について実装、検討した。
この研究ではスクワットという動的な運動に対して骨格推定を用いてトレーニングフォームの改善を促すシステムを提案している。
この研究では理想のスクワットフォームと利用者のスクワットのフォームをフレームごとに分析しその差分を画像上に描画することで可視化している。

また、広島市立大学の岡本らは \cite{Relatedresearch2}陸上のハードル跨ぎの練習においてKinectを用いて骨格を推定し、リアルタイムにフィードバックを返すシステムを提案した。
この研究のシステムはハードル跨ぎ練習でのフォームに対して理想の姿勢との差分をボーンの表示、および文字を画像上に表示することでリアルタイムにフィードバックを返すシステムである。
骨格推定にKinectを用いているのでRGB画像を利用する本研究の骨格推定より精度は高く、深度に関しても取得できるが、Kinectを用意することが必要であり、本研究のシステムよりもコストがかかると考えられる。
