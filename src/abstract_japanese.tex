卒業論文要旨 - 2023年度 (令和05年度)
\begin{center}
\begin{large}
\begin{tabular}{|M{0.97\linewidth}|}
    \hline
      \title \\
    \hline
\end{tabular}
\end{large}
\end{center}

~ \\
  ボディビルをはじめとするフィットネス大会に出場する人は増加傾向にある。日本ボディビル・フィットネス連盟(JBBF)の登録選手数は2015年の2213人から2021年の5576人へと2倍位以上に増加している\cite{jbbf}。
  しかしながら、ボディビル大会への出場は敷居が高く、トレーニング、減量だけでなくステージでの見栄えを良くするためにポージング練習も必須となる。
  ポージング練習は初心者一人で行うのは難しく、トレーナーに指導を受けるという方法があるが高額である。
  本研究では、骨格推定ライブラリであるMediaPipe poseを用いてカメラの入力から理想のポーズとの関節角度を比較し、音声フィードバックを返すシステムを構築した。

~ \\
キーワード:\\
\underline{1. Bodybuilding},
\underline{2. Posing},
\underline{3. Pose estimation},
\begin{flushright}
\dept \\
\author
\end{flushright}
