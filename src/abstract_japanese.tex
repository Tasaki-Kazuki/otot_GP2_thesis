卒業論文要旨 - 2023年度 (令和5年度)
\begin{center}
\begin{large}
\begin{tabular}{|M{0.97\linewidth}|}
    \hline
      \title \\
    \hline
\end{tabular}
\end{large}
\end{center}

~ \\
  ディビルをはじめとするフィットネス大会に出場する人は増加傾向にある。日本ボディビル・フィットネス連盟(JBBF)の登録選手数は2015年の2213人から2021年の5576人へと2倍位以上に増加している\cite{jbbf}。
  しかし、ボディビル競技の成功には、ウェイトトレーニングやポージングスキル、減量などさまざまな重要な要素がある。

  しかし、ポージングは初心者が自己学習するには難易度が高い。初心者がポージングを習得する際の主な障害の一つに1人で練習することが難しいことがある。
  個人トレーナーによる指導は有効だが、多くの場合、費用が高額であり、すべての初心者が利用できるわけではない。
  この問題を解決するために、私は骨格推定技術を用いたポージング練習ツールの開発をした。

  本システムは、MediaPipe Poseという骨格推定ライブラリを用いて、カメラの入力から使用者のポーズを認識し、理想的なポーズとの関節角度の比較を行うシステムである。
  理想のポーズとの関節角度の差をリアルタイムの音声フィードバックを提供し、初心者が1人でスキルを向上させポーズを獲得することを可能にできると考えた。

  実験の結果、本システムを利用することでポーズを改善できた被験者がいたものの統計的に優位とは言えない結果になった。
  鏡を利用した練習方法との比較では本システムと鏡どちらの練習方法でも効果が変わらないであろうことが考えられる結果となった。

  本研究の成果はボディビルポージング練習における初心者の負担軽減につながることが期待できる。

~ \\
キーワード:\\
\underline{1. Bodybuilding},
\underline{2. Posing},
\underline{3. Pose estimation},
\begin{flushright}
\dept \\
\author
\end{flushright}
