卒業論文要旨 - 2023年度 (令和5年度)
\begin{center}
\begin{large}
\begin{tabular}{|M{0.97\linewidth}|}
    \hline
      \title \\
    \hline
\end{tabular}
\end{large} 
\end{center}

~ \\
  ボディビルをはじめとするフィットネス大会に出場する人は増加傾向にある。日本ボディビル・フィットネス連盟(JBBF)の登録選手数は2015年の2213人から2021年の5576人へと2倍位以上に増加している\cite{jbbf}。
  ボディビル競技の成功には、ウェイトトレーニングやポージングスキル、減量などさまざまな重要な要素がある。

  様々な要素の中で特にポージングは初心者が自己学習するには難易度が高い。初心者がポージングを習得する際の主な障害の1つに1人で練習することが難しいことがある。
  その場合、パーソナルトレーナーなど他者よる指導を受けることが考えられるが、多くの場合、費用が高額であり、すべての初心者が利用できるわけではない。
  この問題を解決するために、骨格推定技術を用いたポージング練習ツールの開発をした。

  本システムは、MediaPipe Poseという骨格推定ライブラリを用いて、カメラの入力から利用者のポーズを認識し、理想的なポーズとの関節角度の比較を行うシステムである。
  理想のポーズとの関節角度の差をリアルタイムに音声フィードバックを提供し、初心者が1人でポージングスキルを向上させポーズを獲得することを可能にできると考えた。

  実験の結果、本システムを利用することでポーズを改善できた被験者がいたものの統計的に優位とは言えない結果になった。
  練習方法による比較でも鏡を使った練習に対して統計的に有意な差を示すことができなかったため、サンプル数、練習量を増やすことやシステムの改善を行い、今後検証していきたいと考えている。本システムの有用性を示すことができれば、本研究の成果はボディビルポージング練習における初心者の負担軽減につながることが期待できる。

~ \\
キーワード:\\
\underline{1. Bodybuilding},
\underline{2. Posing},
\underline{3. Pose estimation},
\begin{flushright}
\dept \\
\author
\end{flushright}
