\chapter{本研究における問題定義}
\label{issue}
\section{問題}
\subsection{既存の練習方法}
ボディビルのポージング練習では鏡の前でポーズを取り、視覚的に確認しながらポーズを修正していく方法が一般的である。しかし、初心者では鏡を使った練習ではどこを修正したら良いかわかりづらい。また、鏡を見ながらの練習では左右反転している状態や、視点が自分と同じ高さにあることなどを理由に実際のステージとは異なる見え方をするため、本番を意識したポーズを獲得することが難しい。
既存の鏡の前で自分を見ながら行う練習方法では理想との差異が分かりにくい。鏡の前で行うポージング練習では1人ではポーズを改善することが難しい。

\section{仮説}
理想のフォームとの差異をリアルタイムにフィードバックを行いながらポージング練習を行うことで初心者でも1人で理想のフォームに近づくことができると考える。
%%% Local Variables:
%%% mode: japanese-latex
%%% TeX-master: "./thesis"
%%% End: