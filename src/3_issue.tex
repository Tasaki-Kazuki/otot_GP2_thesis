\chapter{本研究における問題定義}
\label{issue}
\section{問題}
\subsection{既存の練習方法}
ボディビルのポージング練習では鏡の前でポーズを取り、視覚的に確認しながらポーズを修正していく方法が一般的である。しかし、初心者では鏡を使った練習ではどこを修正したら良いかわかりづらい。
また、鏡を見ながらの練習では左右反転している状態や、視点が自分と同じ高さにあることなどを理由に実際のステージ下にいる審査員とは異なる見え方をするため、本番を意識したポーズを獲得することが難しい。
練習方法としては鏡を使った練習や動画、写真に撮ることで見返すという方法がある。しかし、一般的な家庭にあるサイズの鏡ではポーズをとった時に全身が映らなかったり、全身を俯瞰して見ることが難しかったりといった問題がある。
写真や動画に撮る方法では俯瞰して見ることや背中側のポーズを確認するといったことは可能だったが、ポーズをとるごとに見返す手間がある。
ポージングを改善する方法の一つとしてパーソナルトレーニングで専門のトレーナーにフィードバックをもらう方法もある。しかし、パーソナルトレーニングは費用や時間の問題から初心者が何度も通うことはハードルが高い。

\section{仮説}
本研究では、次の仮説を検証する。
\begin{enumerate}
  \item 骨格推定用いた音声フィードバックを用いたポージング練習を行うことでポージングが改善される。
  \item 骨格推定用いた音声フィードバックを用いたポージング練習は鏡を用いたポージング練習と同等以上の効果を出すことができる。
\end{enumerate}
フィードバックがないことが初心者単独でのポージング練習の課題の一つとしてあがる。
骨格推定を用いたフィードバックに従いポーズを修正していくことでポーズが改善されると考えられる。
そして、他者を必要とせずフィードバックを与えることができれば、鏡を用いるポージング練習方法よりも骨格推定を用いたフィードバックを用いたポージング練習の方がポーズ獲得への効果が高いと考える。

