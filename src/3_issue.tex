\chapter{本研究における問題定義}
\label{issue}
\section{問題}
\subsection{既存の練習方法}
ボディビルのポージング練習では鏡の前でポーズを取り、視覚的に確認しながらポーズを修正していく方法が一般的である。しかし、初心者では鏡を使った練習ではどこを修正したら良いかわかりづらい。また、鏡を見ながらの練習では左右反転している状態や、視点が自分と同じ高さにあることなどを理由に実際のステージとは異なる見え方をするため、本番を意識したポーズを獲得することが難しい。\\
また、ポージングを改善する方法の一つとしてパーソナルトレーニングで専門のトレーナーにフィードバックをもらう方法もある。しかし、パーソナルトレーニングは費用や時間の問題から初心者が何度も通うことはハードルが高い。

\section{仮説}
本研究では、次の加越を検証する。
\begin{enumerate}
  \item 骨格推定用いた音声フィードバックを用いたポージング練習を行うことでポージングが改善される。
  \item 骨格推定用いた音声フィードバックを用いたポージング練習は鏡を用いたポージング練習と同等以上の効果を出すことができる。
 \end{enumerate}
